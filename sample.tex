%% 
%% This is a sample doctoral dissertation.  It shows the appropriate
%% structure for your dissertation.  It should handle most of the
%% strange requirements imposed by the Grad school; like the different
%% handling of titles of one/many appendices.  It will automatically
%% handle the linespacing changes.  The body default is double-spaced
%% (except when you use the singlespace or condensed options).  The
%% default for quotations is single-space, and the default for tabular
%% environments is also single-space.  
%%
%% You can change linespacing with the \ls macro.  \ls{1} is
%% single-spaced, \ls{1.5} is for space-and-a-half, and \ls{2} is, you
%% guessed it, double-spaced.
%%
%% This class adds the following commands and environments to the
%% report class, upon which it is based:
%% Commands
%% ------------
%% \ls{n}     -- Sets the linespacing to ``n''
%%
%% \degree{name}{abbrv} -- Sets the name and abbreviation for the degree.
%%                         These default to ``Doctor of Philosopy''
%%                         and ``Ph.D.'', respectively.
%% \copyrightyear{year} -- for the copyright page.
%% \bachelors{degree}{institution} -- for the abstract
%% \masters{degree}{institution}   --  "
%% \committeechair{name}           -- for the signature page
%% \firstreader{name}              --  "
%% \secondreader{name}             --  "
%% \thirdreader{name}              -- (optional)
%% \fourthreader{name}             --  "
%% \fifthreader{name}              --  "
%% \sixthreader{name}              --  "
%% \departmentchair{name}          -- for the signature page
%% \departmentname{name}           --  "
%%
%% \copyrightpage                  -- produces the copyright page
%% \signaturepage                  -- produces the signature page
%%
%% \frontmatter                    -- these are required in their various
%% \mainmatter                     -- appropriate locations
%% \backmatter                     --
%%
%% \unnumberedchapter[toc]{name}   -- like \chapter, except that it
%%                                    produces an unnumbered chapter;
%%                                    alternatively, like \chapter*,
%%                                    except that it lists the chapter
%%                                    in the table of contents.
%%
%% New environments:
%%   dedication  -- for the dedication
%%   abstract    -- for the abstract
%%
%% The thesis documentclass is built on top of the report document class.
%% It accepts all of the options that the report class accepts, plus the
%% following:
%%     doublespace -- the default, indicates double spacing as per U.Mass.
%%                    requirements.  You will need this when you do your
%%                    final copy.
%%     singlespace -- for earlier work, not acceptable to the Grad school
%%     condensed   -- for earlier work, not acceptable to the Grad school,
%%                    creates condensed versions of the frontmatter. 
%%                    Condensed implies singlespace.
%% thesis changes the default font size to 12pt, but you may specify 10pt or
%%   11pt in the options.
\documentclass[singlespace]{thesis}

%%
%% If you have enough figures or tables that you run out of space for their
%% numbers in the List of Tables or List of figures, you can use the following
%% command to adjust the space left for numbers.  The default is shown:
%%
%% \setlength{\tablenumberwidth}{2.3em}

\begin{document}

%%
%% You must fill in all of these appropriately
\title{On the predilection of sheep for grass\\A really long
  title\\with multiple lines}
\author{I. M. Woolly}
\date{February 1995} % The date you'll actually graduate -- must be
                     % February, May, or September
\copyrightyear{1995}
\bachelors{B.Sc.}{University of Woolloomooloo}
\masters{M.Sc.}{University of Back of Bourke}
\committeechair{B. B. Bahh}
\firstreader{Little Bo Peep}
\secondreader{R. U. Sheepish}
\thirdreader{Bill Shepherd}
\fourthreader{Mary Lamb}   % Optional
%\fifthreader{}            % Optional
%\sixthreader{}            % Optional
\departmentchair{Pete Shearer}
\departmentname{Department of Sheep Studies}

%%
%% These lines produce the title, copyright, and signature pages.
%% They are Mandatory; except that you could leave out the copyright page
%% if you were preparing a M.Sc. thesis instead of a PhD dissertation.
\frontmatter
\maketitle
\copyrightpage
\signaturepage

%%
%% Dedication is optional -- but this is how you create it
\begin{dedication}              % Dedication page
  \begin{center}
    \emph{To those little lost sheep.}
  \end{center}
\end{dedication}

%%
%% Epigraph goes here...(Like Acknowledgements???)
%% \chapter{Epigraph}?????

%%
%% Acknowledgements are optional...yeah, right.
\chapter{Acknowledgements}             % Acknowledgements page
  Thanks to all those fine shepherds.

%%
%% Abstract is MANDATORY.
\begin{abstract}                % Abstract
  Sheep like grass.  Why?  Let me tell you...
\end{abstract}

%%
%% Preface goes here...would be just like Acknowledgements
%% \chapter{Preface}
%% ...


%%
%% Table of contents is mandatory, lists of tables and figures are 
%% mandatory if you have any tables or figures; must be in this order.
\tableofcontents                % Table of contents
\listoftables                   % List of Tables
\listoffigures                  % List of Figures

%%
%% We don't handle List of Abbreviations
%% We don't handle Glossary

%%%%%%%%%%%%%%%%%%%%%%%%%%%%%%%%%%%%%%%%%%%%%%%%%%%%%%%%%%%%%%%%%%%%%%%%%
%% Time for the body of the dissertation
\mainmatter   %% <-- This line is mandatory

%%
%% If you want an introduction, which is not a numbered chapter, insert
%% the following two lines.  This is OPTIONAL:
\unnumberedchapter{Introduction}
Why on earth do I want to study sheep anyway?

%%
%% Some sample text
\chapter{An Introduction to Sheep}
Is there life after sheep.  Yes, I say there is.%\marginpar{Really?}

\section{Pulling the wool over your eyes}

Sheep are fabulous creatures.  The noises they make are truly stupendous
\cite{Bah}.  We also want to refer to figure \ref{fig:circle} here.
Here's some verbatim text to screw us up:

{\small
\begin{verbatim}
xxx := y;
xy := x;
\end{verbatim}
}

\begin{figure}
  \begin{center}
    \begin{picture}(300,200)
      \put(150,100){\circle{150}}
      \put(1,1){\framebox(298,198){}}
    \end{picture}
    \caption{A circle in a square.}\label{fig:circle}
  \end{center}
\end{figure}

\subsection{All about sheep noises}
Lots of text here just to fill up some space so we can be sure that we
really are double-spacing and doing all the other things that might be
necessary in formatting a dissertation to U.Mass. guidelines.  We're
also going to have another figure here, figure \ref{fig:disc}, just
for fun, and to make sure that the list of figures is formatted
correctly.  Now it's time for table \ref{table:somenumbers}.  We
really are going to need a third figure, figure \ref{fig:discs}, two
more tables, table \ref{table:morenumbers} and table
\ref{table:evenmorenumbers} and a fourth figure, figure
\ref{fig:circleanddisc}, just to really make sure.

\begin{figure}
  \begin{center}
    \begin{picture}(300,200)
      \put(150,100){\circle*{150}}
      \put(1,1){\framebox(298,198){}}
    \end{picture}
    \caption{A disc in a square.}\label{fig:disc}
  \end{center}
\end{figure}

\begin{table}[htbp]
  \begin{center}
    \caption{Some numbers.}
    \label{table:somenumbers}
    \begin{tabular}{|r|lll|}
      \hline
      & Minimum & Average & Maximum \\
      Type of Animal & Observed & Observed & Observed \\ \hline
      Cats & 12 & 20 & 24 \\
      Dogs & 20 & 20 & 20 \\ \hline
    \end{tabular}
  \end{center}
\end{table}

\begin{figure}
  \begin{center}
    \begin{picture}(400,200)
      \put(100,100){\circle*{150}}
      \put(300,100){\circle*{150}}
      \put(1,1){\framebox(398,198){}}
    \end{picture}
    \caption{Two discs in a rectangle.}\label{fig:discs}
  \end{center}
\end{figure}

\begin{table}[htbp]
  \begin{center}
    \caption{More numbers.}
    \label{table:morenumbers}
    \begin{tabular}{|r|lll|}
      \hline
      Type of Animal & Arms & Legs & Ears \\ \hline
      Person & 2 & 2 & 2 \\
      Dog & 0 & 4 & 2 \\ \hline
    \end{tabular}
  \end{center}
\end{table}

\begin{table}[htbp]
  \begin{center}
    \caption[Even more numbers; together with a caption long enough to ensure that multi-line caption formatting works correctly.]{Even more numbers; together with a caption long enough to ensure that multi-line caption formatting works correctly.  If you want a shorter caption to appear in the Table of Figures you're going to have to put the shorter caption in the \texttt{[]} as shown in this example.}
    \label{table:evenmorenumbers}
    \begin{tabular}{|r|lll|}
      \hline
      x & 1 & 1 & 1 \\ \hline
      y & 2 & 2 & 2 \\
      z & 3 & 3 & 3 \\ \hline
    \end{tabular}
  \end{center}
\end{table}

\begin{figure}
  \begin{center}
    \begin{picture}(400,200)
      \put(100,100){\circle{150}}
      \put(300,100){\circle*{150}}
      \put(1,1){\framebox(398,198){}}
    \end{picture}
    \caption{A circle and a disc in a square.  We want this caption to
      be very long to ensure that the formatting of very long captions
      is handled correctly.  The case of short captions has already
      been dealt with.}\label{fig:circleanddisc}
  \end{center}
\end{figure}



\subsubsection{Baahs}
\subsection{Even more about sheep noises}
\subsection{And yet more about sheep noises}

\section{What about wolves?}

\section{What about shepherds?}
\subsection{A subsection}
\subsubsection{Another subsection}
\paragraph{A Paragraph}
\subparagraph{A Subparagraph}
\subparagraph{Another Subparagraph}
Better not have subparagraphs without text in them.

\chapter{Sheep and Grass}

\section{Introduction}

Grass is a wonderful food...

%% End of body
%%%%%%%%%%%%%%%%%%%%%%%%%%%%%%%%%%%%%%%%%%%%%%%%%%%%%%%%%%%%%%%%%%%%%%%%%%%%%%%

\appendix
\chapter{THE FIRST APPENDIX TITLE}
...
\chapter{THE SECOND APPENDIX TITLE}
...

%%
%% Beginning of back matter
\backmatter  %% <--- mandatory

%%
%% We don't support endnotes

%%
%% A bibliography is required.
\bibliographystyle{thesis}
\bibliography{sample}
\end{document}

%%% Local Variables: 
%%% mode: latex
%%% TeX-master: t
%%% End: 
